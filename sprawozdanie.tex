\documentclass[12pt,a4paper]{article}
\usepackage{siunitx}
\usepackage[utf8]{inputenc}
\usepackage{polski}
\usepackage[polish]{babel}
\usepackage[table,xcdraw]{xcolor}
\usepackage{circuitikz}
\usepackage{graphicx}
\usepackage{listings}
\usepackage{float}
\usepackage{longtable}
\usepackage{gensymb}
\usepackage{hyperref}
\usepackage{multirow}


\lstset{ %
  language=R,                     % the language of the code
  basicstyle=\footnotesize,       % the size of the fonts that are used for the code
  numbers=left,                   % where to put the line-numbers
  numberstyle=\tiny\color{gray},  % the style that is used for the line-numbers
  stepnumber=1,                   % the step between two line-numbers. If it's 1, each line
                                  % will be numbered
  numbersep=5pt,                  % how far the line-numbers are from the code
  backgroundcolor=\color{white},  % choose the background color. You must add \usepackage{color}
  showspaces=false,               % show spaces adding particular underscores
  showstringspaces=false,         % underline spaces within strings
  showtabs=false,                 % show tabs within strings adding particular underscores
  frame=single,                   % adds a frame around the code
  rulecolor=\color{black},        % if not set, the frame-color may be changed on line-breaks within not-black text (e.g. commens (green here))
  tabsize=2,                      % sets default tabsize to 2 spaces
  captionpos=b,                   % sets the caption-position to bottom
  breaklines=true,                % sets automatic line breaking
  breakatwhitespace=false,        % sets if automatic breaks should only happen at whitespace
  title=\lstname,                 % show the filename of files included with \lstinputlisting;
                                  % also try caption instead of title
  keywordstyle=\color{blue},      % keyword style
  commentstyle=\color{dkgreen},   % comment style
  stringstyle=\color{mauve},      % string literal style
  escapeinside={\%*}{*)},         % if you want to add a comment within your code
  morekeywords={*,...}            % if you want to add more keywords to the set
} 

\newcommand{\mymeter}[2] 
{  % #1 = name , #2 = rotation angle
\begin{scope}[transform shape,rotate=#2]
\draw[thick] (#1)node(){$\mathbf V$} circle (11pt);

\end{scope}
}


\newcommand{\newmymeter}[2] 
{  % #1 = name , #2 = rotation angle
\begin{scope}[transform shape,rotate=#2]
\draw[thick] (#1)node(){$\mathbf A$} circle (11pt);

\end{scope}
}



\begin{document}
\begin{titlepage}


\title{%
Sprawozdanie z ćwiczenia laboratoryjnego nr 3.\\
\large  Wyznaczanie $\frac{c_p}{c_v}$ dla powietrza metodą rezonansu akustycznego. Pomiar prędkości dźwięku w powietrzu.}

\author{Piotr Bródka, Ivan Rukhavets, Aliaksandr Sharapa}


\maketitle
\end{titlepage}

\section{Wstęp}
Celem zadania jest wyznaczanie stosunku $c_p$ - ciepła właściwego pod stałym ciśnieniem do $c_v$ - ciepła właściwego w stałej objętości dla powietrza. Stosunek ten oznaczamy w termodynamice przez $\kappa$ i nazywamy stałą adiabaty.
Drugim celem doświadczenia jest też wyznaczenie prędkości rozchodzenia się dźwięku w powietrzu.


\section{Wyznaczanie $\frac{c_p}{c_v}$ dla powietrza metodą rezonansu akustycznego.} 

\subsection{Wstęp teoretyczny}
Przyjęta konwencja jednostek:
\begin{center}

\textit{v} - prędkość

$\rho$ - gęstość


\textit{T} - okres


\textit{f} - częstotliwość


$\bar{T}$ - temperatura


\textit{k} - stała Boltzmana


\end{center}



Są dwie popularne metody wyznaczania $\kappa$:
\begin{enumerate}
\item Clementa-Desormesa, oparta na przeprowadzeniu dwóch przemian gazu: adiabatycznej, a następnie izochorycznej \item Laplace'a, która bazuje na własnościach fal dźwiękowych. 
\end{enumerate} 

To doświadczenie dotyczy drugiej metody. Przedstawimy fakty, które pomogą zrozumieć to doświadczenie.


Fala dźwiękowa jest falą mechaniczną, więc potrzebuje do rozchodzenia się ośrodka (którego cząsteczek drgania są właśnie tą falą).  

W naszym przypadku źródłem tej fali (źródłem drgań) będzie głośnik, a bardziej precyzyjnie - drgania membrany głośnika. 

Fala dźwiękowa jest falą podłużną. Rozchodzi się w powietrzu. Drgania cząsteczek powietrza (naszego ośrodka) objawiają się miejscowym nierównościami ilości cząsteczek na jednostkę objętości, co znaczy: miejscowymi nierównościami ciśnienia.
Możemy przyjąć, że fala dźwiękowa rozchodzi się na tyle szybko, że te zmiany ciśnienia realizowane są w ramach przemiany adiabatycznej gazu (układ nie wymienia ciepła z otoczeniem), czyli spełnione jest: 

$$
pV^\kappa = const.
$$


Ostatecznie:
$$
\kappa = \frac{c_p}{c_v} = \frac{\lambda^2 f^2 M}{k\bar{T}}
$$
lub
$$
\kappa = \frac{c_p}{c_v} = \frac{\lambda^2 M}{k T^2 \bar{T}}
$$


\subsection{Pomiary}
Dokonaliśmy 9 pomiarów dla różnych częstotliwości. Na generatorze ustawialiśmy częstotliwość $f_0$. Generator nie jest jednak dokładny i żeby uzyskać dane do dalszych obliczeń - trzeba było zrobić coś innego. Na oscyloskopie zmierzyliśmy czas jednego pełnego okresu i stąd dostaliśmy prawdziwą częstotliwość. Częstotliwość na generatorze $f_0$ nie ma funkcji innej niż pokazanie, że wartość na generatorze jest obarczona dużym błędem. 

Okres mierzyliśmy przy różnych skali. Błąd obserwatora w przypadku oscyloskopu to pół błędy wzorcowania.

Jest 9 pomiarów dla różnych częstotliwości, rozłożonych w przybliżeniu równomiernie od około 4kHz do około 8kHz. 
Dla każdego z tych pomiarów wyznaczaliśmy wektor (oznaczmy go $w$) kolejnych punktów na linijce, gdy obserwujemy na oscyloskopie rezonans mechaniczny. Taki rezonans występuje co odległość $\frac{1}{2}\lambda$ (połowa długości fali). 

Następnie za pomocą odejmowania $w_{i+1}-w_i$  dostajemy wektor (odnaczmy go $d$) długości połówek fali (w centymetrach). 

Liczymy średnią arymetryczną 

W pomiarze niepewności długości fali mamy błędy:
\begin{enumerate}
\item  typu A: odchylenie standardowe średniej, niepewność średniej $\frac{s_d}{\sqrt{|d|}}$, gdzie $s_d$ to odchylenie standardowe wektora d, a $|d|$ to jego liczność. 
\item typu B: związana z niedokładnościami sprzętu pomiarowego. $$u_B(d) = \sqrt{\frac{(\Delta x)^2}{3} + \frac{(\Delta x_E)^2}{3}}$$
U nas niepewność wzorcowania to długość jednej podziałki, czyli $0.1 cm$, a niepewność eksperymentatora to pół długości jednej podziałki, czyli $0.05 cm$.
Zatem: $u_B(d) = 0.0645497cm $
\end{enumerate} 

Niepewność typu B częstotliwości \textit{f} obliczamy ze wzoru $$\sqrt{\bigg(\frac{\partial \frac{1}{T}}{\partial T}\bigg)^2 U_B(T)^2} = \frac{1}{T^2} \sqrt{\frac{\Delta T^2}{3}+\frac{\Delta T_e^2}{3}}$$

\subsubsection{Pomiar 1}
$$
T = 2.4*0.1ms \hspace{1cm} \Delta T = 0.2*0.1ms \hspace{1cm} \Delta T_e = 0.1*0.1ms 
$$
$$
f_0 = 4023Hz \hspace{1cm} f=4166.7Hz \hspace{1cm} u_B(f)=0.2Hz
$$

w[cm]: 4.8, 9.0, 13.3, 17.5, 21.8, 26.0, 30.3, 34.5, 38.8, 43.0, 47.3

d[cm]: 4.2 4.3 4.2 4.3 4.2 4.3 4.2 4.3 4.2 4.3

\begin{center}
średnia: 4.25cm
\end{center}
$$
u_A(d) = 0.016666667cm
$$

$$u_B(d) = 0.0645497cm $$
$$u_C(d) = 0.06666667cm $$
\subsubsection{Pomiar 2}
$$
T = 4.4*0.05ms \hspace{1cm} \Delta T = 0.2*0.05ms \hspace{1cm} \Delta T_e = 0.1*0.05ms 
$$
$$
f_0 = 4503 Hz \hspace{1cm} f=4545.5Hz \hspace{1cm} u_B(f)=0.1Hz
$$

w[cm]: 0.3, 4.1, 8.0, 11.8, 15.6, 19.5, 23.3, 27.1, 31.0, 34.8, 38.6, 42.5, 46.3

d[cm]: 3.8 3.9 3.8 3.8 3.9 3.8 3.8 3.9 3.8 3.8 3.9 3.8

\begin{center}
średnia: 3.833333cm
\end{center}

$$
u_A(d) = 0.01421338cm
$$
$$u_B(d) = 0.0645497cm $$
$$u_C(d) = 0.06609604cm $$
\subsubsection{Pomiar 3}
$$
T = 4.0*0.05ms \hspace{1cm} \Delta T = 0.2*0.05ms \hspace{1cm} \Delta T_e = 0.1*0.05ms 
$$
$$
f_0 = 4990 Hz \hspace{1cm} f=5000.0Hz \hspace{1cm} u_B(f)=0.2Hz
$$

w[cm]: 2.6, 6.0, 9.6, 13.0, 16.4, 19.8, 23.3, 26.7, 30.2, 33.6, 37.1, 40.6, 44.0, 47.5

d[cm]: 3.4 3.6 3.4 3.4 3.4 3.5 3.4 3.5 3.4 3.5 3.5 3.4 3.5

\begin{center}
średnia: 3.453846cm
\end{center}

$$
u_A(d) = 0.01831135cm
$$
$$u_B(d) = 0.0645497cm $$
$$u_C(d) = 0.06709674cm $$
\subsubsection{Pomiar 4}
$$
T = 3.6*0.05ms \hspace{1cm} \Delta T = 0.2*0.05ms \hspace{1cm} \Delta T_e = 0.1*0.05ms 
$$
$$
f_0 = 5497 Hz \hspace{1cm} f=5556.6Hz \hspace{1cm} u_B(f)=0.2
$$

w[cm]: 3.0, 6.1, 9.2, 12.4, 15.5, 18.7, 21.9, 25.0, 28.1, 31.3, 34.4, 37.6, 40.8, 43.9, 47.1

d[cm]: 3.1 3.1 3.2 3.1 3.2 3.2 3.1 3.1 3.2 3.1 3.2 3.2 3.1 3.2

\begin{center}
średnia: 3.15cm
\end{center}

$$
u_A(d) = 0.0138675cm
$$
$$u_B(d) = 0.0645497cm $$
$$u_C(d) = 0.06602253cm $$
\subsubsection{Pomiar 5}
$$
T = 8.4*0.02ms \hspace{1cm} \Delta T = 0.2*0.02ms \hspace{1cm} \Delta T_e = 0.1*0.02ms 
$$
$$
f_0 = 5986 Hz \hspace{1cm} f=5952.4Hz \hspace{1cm} u_B(f)=0.1Hz
$$

w[cm]: 2.3, 5.2, 8.1, 11.0, 13.9, 16.8, 19.7, 22.6, 25.5, 28.4, 31.3, 34.2, 37.1, 40.0, 42.9, 45.8, 48.7

d[cm]: 2.9 2.9 2.9 2.9 2.9 2.9 2.9 2.9 2.9 2.9 2.9 2.9 2.9 2.9 2.9 2.9

\begin{center}
średnia: 2.9cm
\end{center}

$$
u_A(d) = 0.00cm
$$
$$u_B(d) = 0.0645497cm $$
$$u_C(d) = 0.06454972cm $$
\subsubsection{Pomiar 6}
$$
T = 7.8*0.02ms \hspace{1cm} \Delta T = 0.2*0.02ms \hspace{1cm} \Delta T_e = 0.1*0.02ms 
$$
$$
f_0 = 6502 Hz \hspace{1cm} f=6410.3Hz \hspace{1cm} u_B(f)=0.1Hz
$$

w[cm]: 1.4, 4.0, 6.7, 9.4, 12.1, 14.7, 17.4, 20.0, 22.7, 25.4, 28.0, 30.7, 33.4, 36.0, 38.7, 41.3, 44.0, 46.7, 49.4

d[cm]: 2.6 2.7 2.7 2.7 2.6 2.7 2.6 2.7 2.7 2.6 2.7 2.7 2.6 2.7 2.6 2.7 2.7 2.7

\begin{center}
średnia: 2.666667cm
\end{center}

$$
u_A(d) = 0.01143324cm
$$
$$u_B(d) = 0.0645497cm $$
$$u_C(d) = 0.06555445cm $$
\subsubsection{Pomiar 7}
$$
T = 7.0*0.02ms \hspace{1cm} \Delta T = 0.2*0.02ms \hspace{1cm} \Delta T_e = 0.1*0.02ms 
$$
$$
f_0 = 6995 Hz \hspace{1cm} f=7143.9Hz \hspace{1cm} u_B(f)=0.1Hz
$$

w[cm]: 0.6, 3.2, 5.6, 8.0, 10.5, 13.0, 15.5, 18.0, 20.5, 22.9, 25.4, 27.9, 30.4, 32.8, 35.3, 37.8, 40.3, 42.7, 45.2, 47.7

d[cm]: 2.6 2.4 2.4 2.5 2.5 2.5 2.5 2.5 2.4 2.5 2.5 2.5 2.4 2.5 2.5 2.5 2.4 2.5 2.5

\begin{center}
średnia: 2.478947cm
\end{center}

$$
u_A(d) = 0.0122807cm
$$
$$u_B(d) = 0.0645497cm $$
$$u_C(d) = 0.06570755cm $$
\subsubsection{Pomiar 8}
$$
T = 6.6*0.02ms \hspace{1cm} \Delta T = 0.2*0.02ms \hspace{1cm} \Delta T_e = 0.1*0.02ms 
$$
$$
f_0 = 7504 Hz \hspace{1cm} f=7576.8Hz \hspace{1cm} u_B(f)=0.1Hz
$$

w[cm]: 2.1, 4.4, 6.8, 9.1, 11.4, 13.7, 16.0, 18.3, 20.6, 22.9, 25.2, 27.6, 29.8, 32.2, 34.5, 36.8, 39.1, 41.4, 43.7, 46.0, 48.3

d[cm]: 2.3 2.4 2.3 2.3 2.3 2.3 2.3 2.3 2.3 2.3 2.4 2.2 2.4 2.3 2.3 2.3 2.3 2.3 2.3 2.3

\begin{center}
srednia: 2.31cm
\end{center}

$$
u_A(d) = 0.01cm
$$
$$u_B(d) = 0.06454972cm $$
$$u_C(d) = 0.06531973cm $$
\subsubsection{Pomiar 9}
$$
T = 6.2*0.02ms \hspace{1cm} \Delta T = 0.2*0.2ms \hspace{1cm} \Delta T_e = 0.1*0.02ms 
$$
$$
f_0 = 7988Hz \hspace{1cm} f=8065.5Hz \hspace{1cm} u_B(f)=0.2Hz
$$

w[cm]: 1.0, 3.2, 5.3, 7.6, 9.8, 11.9, 14.1, 16.3, 18.4, 20.6, 22.8, 25.0, 27.2, 29.3, 31.4, 33.6, 35.1, 37.9, 40.1, 42.3, 44.5, 46.6, 48.8

d[cm]: 2.2 2.1 2.3 2.2 2.1 2.2 2.2 2.1 2.2 2.2 2.2 2.2 2.1 2.1 2.2 1.5 2.8 2.2 2.2 2.2 2.1 2.2

\begin{center}

średnia: 2.172727
\end{center}

$$
u_A(d) = 0.04422582
$$
$$u_B(d) = 0.0645497cm $$
$$u_C(d) = 0.07824698cm $$

Pod koniec zmierzyliśmy temperaturę która wyniosła $21^\circ C = 294.15 K$
\subsection{Wyliczenie $\frac{c_p}{c_v}$}

Dla każdego z 9 pomiarów wyznaczamy $\kappa=\frac{c_p}{c_v}$ ze wzoru 
$$
\kappa = \frac{c_p}{c_v} = \frac{(2\bar{d})^2 f^2 M}{k\bar{T}}
$$

Zapiszemy wyniki wszystkich pomiarów do tablicy

\begin{table}[H]
\centering

\label{my-label}
\begin{tabular}{|l|l|l|l|l|l|l|l|}
\hline
Lp. & k{[}J/K{]} *$10^{-23}$       & M{[}Kg{]} *$10^{-26}$      & T{[}K{]}                & $f[s^{-1}]$ & $\Delta f[s^{-1}]$ & $\lambda[cm]$ & $\kappa$ \\ \hline
1   & \multirow{9}{*}{1.3806} & \multirow{9}{*}{4.81} & \multirow{9}{*}{291.15} & 4167.7     & 0.2         & 9.5           & 1.49  \\ \cline{1-1} \cline{5-8} 
2   &                         &                       &                         & 4545.5     & 0.1         & 7.66           & 1.44  \\ \cline{1-1} \cline{5-8} 
3   &                         &                       &                         & 5000.0     & 0.2         & 6.90           & 1.41  \\ \cline{1-1} \cline{5-8} 
4   &                         &                       &                         & 5556.6     & 0.2         & 6.30           & 1.45  \\ \cline{1-1} \cline{5-8} 
5   &                         &                       &                         & 5952.4     & 0.1         & 5.80          & 1.41  \\ \cline{1-1} \cline{5-8} 
6   &                         &                       &                         & 6410.3     & 0.1         & 5.34          & 1.38  \\ \cline{1-1} \cline{5-8} 
7   &                         &                       &                         & 7143.9     & 0.1         & 4.94          & 1.49  \\ \cline{1-1} \cline{5-8} 
8   &                         &                       &                         & 7576.8     & 0.1         & 4.62           & 1.45  \\ \cline{1-1} \cline{5-8} 
9   &                         &                       &                         & 8065.5     & 0.2         & 4.34           & 1.45  \\ \hline
\end{tabular}
\caption{Wyniki pomiarów i obliczone $\kappa$}
\end{table}
Średnia wartość $\kappa$ wyniosła 1.4417
Obliczamy błąd średni kwadratowy średniej arytmetycznej $kappa$ ze wzoru
$$
\Delta \kappa = \sqrt{\frac{\sum\limits_{i=1}^{n}{(\kappa-\kappa_i)^2}}{n(n-1)}} =  0.011
$$

Mamy 
$$
\kappa = 1.4417 \pm 0.011
$$

\section{Pomiar prędkości dźwięku w powietrzu.}



\end{document}