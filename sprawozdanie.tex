\documentclass[12pt,a4paper]{article}
\usepackage{siunitx}
\usepackage[utf8]{inputenc}
\usepackage{polski}
\usepackage[polish]{babel}
\usepackage[table,xcdraw]{xcolor}
\usepackage{circuitikz}
\usepackage{graphicx}
\usepackage{listings}
\usepackage{float}
\usepackage{longtable}
\usepackage{gensymb}
\usepackage{hyperref}
\usepackage{adjustbox,lipsum}

\lstset{ %
  language=R,                     % the language of the code
  basicstyle=\footnotesize,       % the size of the fonts that are used for the code
  numbers=left,                   % where to put the line-numbers
  numberstyle=\tiny\color{gray},  % the style that is used for the line-numbers
  stepnumber=1,                   % the step between two line-numbers. If it's 1, each line
                                  % will be numbered
  numbersep=5pt,                  % how far the line-numbers are from the code
  backgroundcolor=\color{white},  % choose the background color. You must add \usepackage{color}
  showspaces=false,               % show spaces adding particular underscores
  showstringspaces=false,         % underline spaces within strings
  showtabs=false,                 % show tabs within strings adding particular underscores
  frame=single,                   % adds a frame around the code
  rulecolor=\color{black},        % if not set, the frame-color may be changed on line-breaks within not-black text (e.g. commens (green here))
  tabsize=2,                      % sets default tabsize to 2 spaces
  captionpos=b,                   % sets the caption-position to bottom
  breaklines=true,                % sets automatic line breaking
  breakatwhitespace=false,        % sets if automatic breaks should only happen at whitespace
  title=\lstname,                 % show the filename of files included with \lstinputlisting;
                                  % also try caption instead of title
  keywordstyle=\color{blue},      % keyword style
  commentstyle=\color{dkgreen},   % comment style
  stringstyle=\color{mauve},      % string literal style
  escapeinside={\%*}{*)},         % if you want to add a comment within your code
  morekeywords={*,...}            % if you want to add more keywords to the set
} 

\newcommand{\mymeter}[2] 
{  % #1 = name , #2 = rotation angle
\begin{scope}[transform shape,rotate=#2]
\draw[thick] (#1)node(){$\mathbf V$} circle (11pt);

\end{scope}
}


\newcommand{\newmymeter}[2] 
{  % #1 = name , #2 = rotation angle
\begin{scope}[transform shape,rotate=#2]
\draw[thick] (#1)node(){$\mathbf A$} circle (11pt);

\end{scope}
}



\begin{document}
\begin{titlepage}


\title{%
Sprawozdanie z ćwiczenia laboratoryjnego nr 3.\\
\large  Wyznaczanie $\frac{c_p}{c_v}$ dla powietrza metodą rezonansu akustycznego. Pomiar prędkości dźwięku w powietrzu.}

\author{Piotr Bródka, Ivan Rukhavets, Aliaksandr Sharapa}


\maketitle
\end{titlepage}

\section{Wstęp}
Celem zadania jest wyznaczanie stosunku $c_p$ - ciepła właściwego pod stałym ciśnieniem do $c_v$ - ciepła właściwego w stałej obiętości dla powietrza. Stosunek ten oznaczamy w termodynamice przez $\kappa$ i nazywamy stałą adiabaty.
Drugim celem doświadczenia jest też wyznaczenie prędkości rozchodzenia się dźwięku w powietrzu.


\section{Wyznaczanie $\frac{c_p}{c_v}$ dla powietrza metodą rezonansu akustycznego.} 

\subsection{Wstęp teoretyczny}
Przyjęta konwencja jednostek:
$$
v - \text{prędkość}
$$
$$
\rho - \text{gęstość}
$$
$$
T - \text{okres}
$$
$$
f - \text{częstotliwość}
$$
$$
\bar{T} - \text{temperatura}
$$
$$
k - \text{stała Boltzmana}
$$



Są dwie popularne metody wyznaczania $\kappa$:
\begin{enumerate}
\item Clementa-Desormesa, oparta na przeprowadzeniu dwóch przemian gazu: adiabatycznej, a następnie izochorycznej \item Laplace'a, która bazuje na własnościach fal dźwiękowych. 
\end{enumerate} 

To doświadczenie dotyczy drugiej metody. Przedstawimy fakty, które pomogą zrozumieć to doświadczenie.


Fala dźwiękowa jest falą mechaniczną, więc potrzebuje do rozchodzenia się ośrodka (którego cząsteczek drgania są właśnie tą falą).  

W naszym przypadku źródłem tej fali (źródłem drgań) będzie głośnik, a bardziej precyzyjnie - drgania membrany głośnika. 

Fala dźwiękowa jest falą podłużną. Rozchodzi się w powietrzu. Drgania cząsteczek powietrza (naszego ośrodka) objawiają się miejscowym nierównościami ilości cząsteczek na jednostkę objętości, co znaczy: miejscowymi nierównościami ciśnienia.
Możemy przyjąć, że fala dźwiękowa rozchodzi się na tyle szybko, że te zmiany ciśnienia realizowane są w ramach przemiany adiabatycznej gazu (układ nie wymiania ciepła z otoczeniem), czyli spełnione jest: 

$$
pV^\kappa = const.
$$


Ostatecznie:
$$
\kappa = \frac{c_p}{c_v} = \frac{\lambda^2 f^2 M}{k\bar{T}}
$$
lub
$$
\kappa = \frac{c_p}{c_v} = \frac{\lambda^2 M}{k T^2 \bar{T}}
$$


\subsection{Pomiary}
Dokonaliśmy 9 pomiarów dla różnych częstotliwości. Na generatorze ustawialiśmy częstotliwość $f_0$. Generator nie jest jednak dokładny i żeby uzyskać dane do dalszych obliczeń - trzeba było zrobić coś innego. Na oscyloskopie zmierzyliśmy czas jednego pełnego okresu i stąd dostaliśmy prawdziwą częstotliwość. Częstotliwość na generatorze $f_0$ nie ma funkcji innej niż pokazanie, że wartość na generatorze jest obarczona dużym błędem. 

Okres mierzyliśmy przy różnych skali. Błąds obserwatora w przypadku oscyloskopu to pół błędy wzorcowania.

Jest 9 pomiarów dla różnych częstotliwości, rozłożonych w przybliżeniu równomiernie od około 4kHz do około 8kHz. 
Dla każdego z tych pomiarów wyznaczaliśmy wektor (oznaczmy go $w$) kolejnych punktów na linijce, gdy obserwujemy na oscyloskopie rezonans mechaniczny. Taki rezonans występuje co odległość $\frac{1}{2}\lambda$ (połowa długości fali). 

Następnie za pomocą odejmowania $w_{i+1}-w_i$  dostajemy wektor (odnaczmy go $d$) długości połówek fali (w centymetrach). 

Liczymy średnią arytmetyczną 

W pomiarze niepewności długości fali mamy błędy:
\begin{enumerate}
\item  typu A: odchylenie standardowe średniej, niepewność średniej $\frac{s_d}{\sqrt{|d|}}$, gdzie $s_d$ to odchylenie standardowe wektora d, a $|d|$ to jego liczność. 
\item typu B: związana z niedokładnościami sprzętu pomiarowego. $$u_b = \sqrt{\frac{(\Delta x)^2}{3} + \frac{(\Delta x_E)^2}{3}}$$
U nas niepewność wzorcowania to długość jednej podziałki, czyli $0.1 cm$, a niepewność eksperymentatora to pół długości jednej podziałki, czyli $0.05 cm$.
Zatem: $u_b = 0.0645497cm $
\end{enumerate} 

\subsubsection{Pomiar 1}
$$
T = 2.4*0.1ms \hspace{1cm} \Delta T = 0.2*0.1ms \hspace{1cm} \Delta T_e = 0.1*0.1ms 
$$
$$
f_0 = 4023Hz \hspace{1cm} f=4167Hz
$$

\begin{center}
\begin{tabular}{|c|ccccccccccc|}
\hline 
$w[cm]$ & 4.8& 9.0& 13.3& 17.5& 21.8& 26.0& 30.3& 34.5& 38.8& 43.0& 47.3 \\ 
\hline 
$\Delta[cm]$ & & 4.2& 4.3& 4.2& 4.3& 4.2& 4.3& 4.2& 4.3& 4.2& 4.3 \\ 
\hline 
\end{tabular} 
\end{center}


$$\text{średnia}: 4.25cm$$
$$u_a = 0.017cm \hspace{0.5cm} u_b = 0.065cm \hspace{0.5cm} u_c = 0.067cm $$

\subsubsection{Pomiar 2}
$$
T = 4.4*0.05ms \hspace{1cm} \Delta T = 0.2*0.05ms \hspace{1cm} \Delta T_e = 0.1*0.05ms 
$$
$$
f_0 = 4503 Hz \hspace{1cm} f=4545Hz
$$

\begin{center}
\begin{tabular}{|c|ccccccccccccc|}
\hline 
$w[cm]$ & 0.3 & 4.1 & 8.0& 11.8& 15.6& 19.5& 23.3& 27.1& 31.0 &34.8& 38.6 &42.5& 46.3 \\ 
\hline 
$\Delta[cm]$ & & 3.8& 3.9& 3.8& 3.8& 3.9& 3.8& 3.8& 3.9& 3.8 &3.8 &3.9& 3.8 \\ 
\hline 
\end{tabular} 
\end{center}

$$\text{średnia}: 3.83cm$$
$$u_a = 0.014cm \hspace{0.5cm} u_b = 0.065cm \hspace{0.5cm} u_c = 0.066cm $$

\subsubsection{Pomiar 3}
$$
T = 4.0*0.05ms \hspace{1cm} \Delta T = 0.2*0.05ms \hspace{1cm} \Delta T_e = 0.1*0.05ms 
$$
$$
f_0 = 4990 Hz \hspace{1cm} f=5000Hz
$$

\begin{adjustbox}{max width=1.2\textwidth}
\begin{tabular}{|c|cccccccccccccc|}
\hline 
$w[cm]$ & 2.6 & 6.0 & 9.6 &13.0& 16.4& 19.8 &23.3& 26.7 &30.2& 33.6 &37.1 &40.6& 44.0& 47.5 \\ 
\hline 
$\Delta[cm]$ & & 3.4 &3.6& 3.4& 3.4 &3.4 &3.5& 3.4 &3.5& 3.4& 3.5& 3.5& 3.4& 3.5 \\ 
\hline 
\end{tabular} 
\end{adjustbox}


$$\text{średnia}: 3.45cm$$
$$u_a = 0.018cm \hspace{0.5cm} u_b = 0.065cm \hspace{0.5cm} u_c = 0.067cm $$

\subsubsection{Pomiar 4}
$$
T = 3.6*0.05ms \hspace{1cm} \Delta T = 0.2*0.05ms \hspace{1cm} \Delta T_e = 0.1*0.05ms 
$$
$$
f_0 = 5497 Hz \hspace{1cm} f=5556Hz
$$

\begin{adjustbox}{max width=1.2\textwidth}
\begin{tabular}{|c|ccccccccccccccc|}
\hline 
$w[cm]$ & 3.0 & 6.1 & 9.2& 12.4& 15.5& 18.7 &21.9& 25.0& 28.1 &31.3& 34.4& 37.6& 40.8& 43.9& 47.1 \\ 
\hline 
$\Delta[cm]$ & & 3.1& 3.1& 3.2& 3.1& 3.2 &3.2 &3.1& 3.1& 3.2 &3.1& 3.2& 3.2& 3.1& 3.2 \\ 
\hline 
\end{tabular} 
\end{adjustbox}

$$\text{średnia}: 3.15cm$$
$$u_a = 0.014cm \hspace{0.5cm} u_b = 0.065cm \hspace{0.5cm} u_c = 0.066cm $$

\subsubsection{Pomiar 5}
$$
T = 8.4*0.02ms \hspace{1cm} \Delta T = 0.2*0.02ms \hspace{1cm} \Delta T_e = 0.1*0.02ms 
$$
$$
f_0 = 5986 Hz \hspace{1cm} f=5952Hz
$$

\begin{adjustbox}{max width=1.2\textwidth}
\begin{tabular}{|c|ccccccccccccccccc|}
\hline 
$w[cm]$ &  2.3 & 5.2&  8.1 &11.0& 13.9& 16.8& 19.7& 22.6& 25.5& 28.4 &31.3& 34.2& 37.1& 40.0& 42.9 &45.8& 48.7 \\ 
\hline 
$\Delta[cm]$ & & 2.9 &2.9& 2.9 &2.9& 2.9& 2.9& 2.9 &2.9& 2.9& 2.9& 2.9& 2.9 &2.9& 2.9& 2.9& 2.9 \\ 
\hline 
\end{tabular} 
\end{adjustbox}

$$\text{średnia}: 2.90cm$$
$$u_a = 0.000cm \hspace{0.5cm} u_b = 0.065cm \hspace{0.5cm} u_c = 0.065cm $$

\subsubsection{Pomiar 6}
$$
T = 7.8*0.02ms \hspace{1cm} \Delta T = 0.2*0.02ms \hspace{1cm} \Delta T_e = 0.1*0.02ms 
$$
$$
f_0 = 6502 Hz \hspace{1cm} f=6410Hz
$$

\begin{adjustbox}{max width=1.2\textwidth}
\begin{tabular}{|c|ccccccccccccccccccc|}
\hline 
$w[cm]$ & 1.4 & 4.0 & 6.7 &9.4 &12.1& 14.7 &17.4 &20.0& 22.7 &25.4 &28.0& 30.7& 33.4 &36.0& 38.7& 41.3& 44.0& 46.7& 49.4\\ 
\hline 
$\Delta[cm]$ & & 2.6& 2.7& 2.7& 2.7& 2.6 &2.7 &2.6 &2.7 &2.7 &2.6 &2.7& 2.7& 2.6& 2.7& 2.6& 2.7& 2.7& 2.7 \\ 
\hline 
\end{tabular} 
\end{adjustbox}

$$\text{średnia}: 2.67cm$$
$$u_a = 0.011cm \hspace{0.5cm} u_b = 0.065cm \hspace{0.5cm} u_c = 0.066cm $$

\subsubsection{Pomiar 7}
$$
T = 7.0*0.02ms \hspace{1cm} \Delta T = 0.2*0.02ms \hspace{1cm} \Delta T_e = 0.1*0.02ms 
$$
$$
f_0 = 6995 Hz \hspace{1cm} f=7143Hz
$$

\begin{adjustbox}{max width=1.2\textwidth}
\begin{tabular}{|c|cccccccccccccccccccc|}
\hline 
$w[cm]$ & 0.6 & 3.2 & 5.6 & 8.0& 10.5 &13.0 &15.5 &18.0 &20.5& 22.9 &25.4& 27.9& 30.4 &32.8 &35.3 &37.8& 40.3 &42.7& 45.2 &47.7\\ 
\hline 
$\Delta[cm]$ & & 2.6 &2.4& 2.4& 2.5& 2.5 &2.5 &2.5& 2.5 &2.4 &2.5 &2.5& 2.5& 2.4& 2.5& 2.5& 2.5& 2.4& 2.5& 2.5 \\ 
\hline 
\end{tabular} 
\end{adjustbox}

$$\text{średnia}: 2.48cm$$
$$u_a = 0.012cm \hspace{0.5cm} u_b = 0.065cm \hspace{0.5cm} u_c = 0.066cm $$

\subsubsection{Pomiar 8}
$$
T = 6.6*0.02ms \hspace{1cm} \Delta T = 0.2*0.02ms \hspace{1cm} \Delta T_e = 0.1*0.02ms 
$$
$$
f_0 = 7504 Hz \hspace{1cm} f=7576Hz
$$

\begin{adjustbox}{max width=1.2\textwidth}
\begin{tabular}{|c|ccccccccccccccccccccc|}
\hline 
$w[cm]$ & 2.1 & 4.4 & 6.8  &9.1& 11.4& 13.7 &16.0& 18.3& 20.6 &22.9& 25.2& 27.6& 29.8& 32.2& 34.5& 36.8& 39.1& 41.4& 43.7& 46.0& 48.3\\ 
\hline 
$\Delta[cm]$ & & 2.3& 2.4& 2.3& 2.3& 2.3 &2.3 &2.3 &2.3& 2.3& 2.3& 2.4& 2.2& 2.4& 2.3& 2.3& 2.3& 2.3& 2.3 &2.3& 2.3 \\ 
\hline 
\end{tabular} 
\end{adjustbox}

$$\text{średnia}: 2.31cm$$
$$u_a = 0.010cm \hspace{0.5cm} u_b = 0.065cm \hspace{0.5cm} u_c = 0.065cm $$

\subsubsection{Pomiar 9}
$$
T = 6.2*0.02ms \hspace{1cm} \Delta T = 0.2*0.02ms \hspace{1cm} \Delta T_e = 0.1*0.02ms 
$$
$$
f_0 = 7988Hz \hspace{1cm} f=8065Hz
$$

\begin{adjustbox}{max width=1.2\textwidth}
\begin{tabular}{|c|ccccccccccccccccccccccc|}
\hline 
$w[cm]$ & 1.0 & 3.2 & 5.3&  7.6 & 9.8& 11.9 &14.1& 16.3 &18.4 &20.6 &22.8& 25.0& 27.2& 29.3 &31.4 &33.6 &35.1& 37.9 &40.1& 42.3& 44.5& 46.6 &48.8\\ 
\hline 
$\Delta[cm]$ & & 2.2& 2.1& 2.3 &2.2& 2.1& 2.2& 2.2& 2.1 &2.2& 2.2& 2.2 &2.2& 2.1& 2.1& 2.2& 1.5 &2.8& 2.2& 2.2 &2.2& 2.1& 2.2 \\ 
\hline 
\end{tabular} 
\end{adjustbox}

$$\text{średnia}: 2.17cm$$
$$u_a = 0.044cm \hspace{0.5cm} u_b = 0.065cm \hspace{0.5cm} u_c = 0.078cm $$


\section{Pomiar prędkości dźwięku w powietrzu.}

\end{document}